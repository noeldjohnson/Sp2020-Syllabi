\documentclass[11pt]{article}
\usepackage{graphicx}
\usepackage{float}
\usepackage{mathrsfs}
\usepackage{amsfonts}
\usepackage{footmisc}
\usepackage{amsmath}
\usepackage{amssymb}
\usepackage{parskip}
\usepackage{appendix}
\usepackage{natbib}\setlength{\bibsep}{0.2pt}
\usepackage{caption}
\usepackage{fullpage}
\usepackage{setspace}
\usepackage{bibentry}
\usepackage{enumerate}
\pagestyle{empty}
\usepackage[hyphens]{url}
\usepackage{hyperref} 
\renewcommand{\thefootnote}{\fnsymbol{footnote}}

\begin{document}
\nobibliography*

\begin{center}
{\bf Econ. 360:  Economics of Developing Areas \\
George Mason University, Spring 2020 \\
W 7:20 - 10:00 PM,  Room: Robinson Hall B111}
\end{center}

\setlength{\unitlength}{1in}

\begin{picture}(6,.1) 
\put(0,0) {\line(1,0){6.25}}         
\end{picture}

 

\renewcommand{\arraystretch}{2}

\vskip.15in
\noindent\textbf{Instructor:} Noel D. Johnson
\vskip.15in
\noindent\textbf{Email:} \href{mailto:noeldjohnson@mac.com}{noeldjohnson@mac.com}
\vskip.15in
\noindent\textbf{Office Location:} Carow 8
\vskip.15in
\noindent\textbf{Office Hours:} TBA
\vskip.15in
\noindent\textbf{Course Webpage:} \url{https://noeldjohnson.github.io/dev/}

\vskip.15in
\noindent \textbf{Course Description}

The aim of this course is to introduce various ideas and approaches to understanding economic growth. The course is split, roughly, into four sections. In the first, we will review the basic facts of economic growth as well as discuss current empirical techniques used to study growth. In the second section we will discuss factor accumulation as an explanation for growth. In the third section we will discuss various explanations for differences in productivity across regions. In the final section we will discuss deep determinants of economic growth. Throughout the course an effort will be made to link the theory of economic growth with empirical evidence. Textbook readings will be supplemented with recent academic papers and blog posts.


\vspace{.5cm}

\begin{tabbing}

\textbf{Week 1} \hspace{1.5cm} \=Introduction \\
\\
\textbf{Week 2}   \>Capital's Role in Production \\
\\
\textbf{Week 3}  \>The Role of Population in Economic Growth \\
\\
\textbf{Week 4}  \>The Effect of Human Capital on Growth \\
\\
\textbf{Week 5}   \>Measuring Productivity Growth Across Countries \\
\\
\textbf{Week 6}   \>Technology and Productivity \\
\\
\textbf{Week 7}   \>Inefficiency and Productivity \\
\\
\textbf{Week 8}   \>Catch-up and Review for Midterm \\
\\
\textbf{Week 9}   \>Midterm Exam \\
\\
\textbf{Week 10}   \>Government \\
\\
\textbf{Week 11}  \>Inequality \\
\\
\textbf{Week 12}  \>Culture \\
\\
\textbf{Week 13}  \>Geography, Climate, and Natural Resources \\
\\
\textbf{Week 14}   \>Review of Material for Final Exam \\
\\


\end{tabbing}


\vskip.15in
\noindent\textbf{Course Requirements}

\vskip.15in
\textit{Eighty percent of success is showing up} -- Woody Allen \\
\vskip.15in


Your course grade will be based on the following:

\begin{enumerate}

\item  A midterm exam worth 30\% of your grade.  The date of the midterm is approximately placed in the course outline. If you miss the midterm, then you may not ``re-take'' it.  The weight of your final will be increased accordingly.

\item Problem sets worth 15\% of your grade.

\item  Book Essay worth 25\% of your grade.

\item  A comprehensive final exam worth 30\% of your grade.  The final will be given during the university assigned final exam time.  If you miss the Final without a university sanctioned excuse then you will receive an F for the course.

\end{enumerate}

\vskip.35in
\noindent\textbf{Grade Disputes}

If you wish to dispute the grading of an exam you must submit a typed request making explicit reference to the problem(s) along with the original test within two class periods (one week) of the day I hand back the graded exams. I will then review your arguments and decide if a mistake was made. Under no circumstances will I discuss grading with a student until after I have received and reviewed the written complaint. Office hours are for me to help you understand course material, not for grade disputes.

\vskip.35in
\noindent\textbf{Book Essay}

You are required to write a book essay worth 25\% of your grade. You must choose two books from the list below and write an essay describing how these books relate to the themes we discussed in class. The paper should start with an introduction that lays out what books you have chosen and briefly describes their main hypotheses. You should then spend a significant portion of the paper overviewing the main arguments of the books. Please explain what the arguments are and how the author(s) substantiate the arguments. How do these arguments relate to the theory or other readings we've done in class? Do they complement each other or are they critiques?

The paper should be typed in a reasonable font. Double spaced with reasonable margins. The paper should be around 10 to 15 pages. The paper is due on the last day of class---6 May---by midnight. \textbf{I will only accept pdf's emailed to me}. For every 24 hours the paper is late, your paper grade will be reduced by one half of a letter grade (i.e. 5\%).
\bigskip


\begin{itemize}
\item \bibentry{gidla2017ants}
\smallskip
\item \bibentry{kuran2012long}
\smallskip
\item \bibentry{scheidel2018great}
\smallskip
\item \bibentry{allen2009british}
\smallskip
\item \bibentry{mokyr2010enlightened}
\smallskip
\item \bibentry{rubin2017rulers}
\smallskip
\item \bibentry{johnson2019persecution}
\smallskip
\item \bibentry{acemoglu2019narrow}
\smallskip
\item \bibentry{banerjee2019good}
\smallskip
\item \bibentry{deaton2013great}
\smallskip
\item \bibentry{easterly2002elusive}
\smallskip
\item \bibentry{collins2010portfolios}
\smallskip
\item \bibentry{wrong2009s}
\smallskip
\item \bibentry{sen2001development}
\smallskip
\item \bibentry{north2009violence}
\smallskip
\item \bibentry{mcneill1998plagues}
\smallskip
\item \bibentry{diamond1998guns}
\end{itemize}


\vskip.35in
\noindent\textbf{Class Attendance and Participation}

Participation is important for this class and you should attend class unless you are ill. I will hand out sign-up sheets on randomly chosen dates to evaluate attendance. You can miss one class for any reason. Beyond one class, each recorded absence will negatively impact your final grade by 2.5-percentage points.


\vskip.35in
\noindent\textbf{Course Materials}

There is one required text: David N. Weil, Economic Growth, 3rd edition. Copies have been ordered at the GMU Bookstore.

We will also be reading  articles which are available for download on  from either JSTOR, the NBER Working Papers archive, or EconLit through the GMU libraries research databases page.  You are required to acquire these papers and read them before the class in which they are covered.  You are expected to know the readings for the exams.

\vskip.15in
\noindent\textbf{Some Important Dates}

First Day of Classes:  22 January

Spring Break: 9 March to 15 March

Last Day of Classes:  29 April

Final Exam: 6 May from 7:30 pm--10:15 pm

\newpage

\vskip.15in
\noindent\textbf{PLEASE NOTE: COURSE POLICIES}
 
\textbf{1. George Mason University Honor System and Code}

\textit{Honor Code}

George Mason University has an Honor Code, which requires all members of this community to maintain the highest standards of academic honesty and integrity. Cheating, plagiarism, lying, and stealing are all prohibited.
 
All violations of the Honor Code will be reported to the Honor Committee.

Plagiarism (statements from Mason Web Site)

Plagiarism means using the exact words, opinions, or factual information from another person without giving that person credit.

http://mason.gmu.edu/~montecin/plagiarism.htm\#plagiarism
 
Please familiarize yourself with the Honor System and Code, as stated in the George Mason University Undergraduate Catalog. When you are given an assignment as an individual, the work must be your own. Some of your work may be collaborative; source material for group projects and work of individual group members must be carefully documented for individual contributions.

http://mason.gmu.edu/~montecin/plagiarism.htm
 
\textbf{2. Class Registration}
 
Students are responsible for verifying the accuracy of their own schedules. Students need to check PatriotWeb regularly to verify that they are registered for the classes that they think they are. This is particularly important since students are no longer dropped for nonpayment. Faculty may not allow a student who is not registered to continue to attend class and may not grade the work of students who do not appear on the official class roster.

Deadlines each semester are published in the Schedule of Classes available from the Registrar's Web Site registrar.gmu.edu

After the last day to drop a class, withdrawing from this class requires the approval of the dean and is only allowed for nonacademic reasons. Undergraduate students may choose to exercise a selective withdrawal. See the Schedule of Classes for selective withdrawal procedures.

\textbf{3. Accommodations for students with disabilities:}
 
If you are a student with a disability and you need academic accommodations, please see me and contact the Office of Disability Resources at 703-993-2474. All academic accommodations must be arranged through that office.
 
The need for accommodations should be identified at the beginning of the semester and the specific accommodation has to be arranged through the Office of Disability Resources. Faculty cannot provide accommodations to students on their own (e.g. allowing a student extra time to complete an exam because the student reports having a disability).

\newpage
\noindent \textbf{Course Outline} (subject to change)
\vskip.15in


\noindent \textbf{\textit{Week 1: Introduction}}
\begin{itemize}
\item Weil Chapters 1 and 2: Differences in the level and rate of income growth among countries
\smallskip
\item Chapter 12 in \bibentry{hartmann1983quiet}
\smallskip
\item \href{https://goo.gl/ZwYffc}{Hans Rosling BBC video on income differences over time}
\smallskip
\item MR University \href{https://goo.gl/ks5c5g}{``When in India, Get a Haircut''}
\smallskip
\item Pages 291 to 300 of \bibentry{freedman1991statistical}
\smallskip
\end{itemize}
\bigskip

\noindent \textbf{\textit{Week 2: Capital's Role in Production}}
\begin{itemize}
\item Weil Chapter 3: Capital's Role in Production
\smallskip
\item \bibentry{pritchett1997divergence}
\smallskip
\item \textbf{Problem Set 1 Distributed}
\end{itemize}
\bigskip

\noindent \textbf{\textit{Week 3: The Role of Population in Economic Growth}}
\begin{itemize}
\item Weil Chapter 4: The Role of Population in Economic Growth (skip appendix).
\smallskip
\item Pages 681 to 687 of \bibentry{kremer1993population}
\smallskip
\end{itemize}
\bigskip

\noindent \textbf{\textit{Week 4: The Effect of Human Capital on Growth}}
\begin{itemize}
\item Weil Chapter 6: The Effect of Human Capital on Growth
\smallskip
\item \bibentry{bleakley2007disease}
\smallskip
\item \textbf{Problem Set 2 Distributed}
\end{itemize}
\bigskip

\noindent \textbf{\textit{Week 5: How much does productivity growth differ among countries?}}
\begin{itemize}
\item Weil Chapter 7: How much does productivity growth differ among countries?
\smallskip
\item \textbf{Problem Set 1 Collected}
\end{itemize}
\bigskip

\noindent \textbf{\textit{Week 6: Technology and Productivity}}
\begin{itemize}
\item Weil Chapters 8 and 9: The Role of Technology in Growth
\smallskip
\item Joel Mokyr \href{https://www.theatlantic.com/business/archive/2016/11/progress-isnt-natural-mokyr/507740/}{``Progress Isn't Natural''}
\smallskip
\item Freakonomics Podcast \href{http://freakonomics.com/podcast/no-new-ideas/}{``Are We Running Out of Ideas?''}
\end{itemize}
\bigskip

\noindent \textbf{\textit{Week 7: Inefficiency and Productivity}}
\begin{itemize}
\item Weil Chapter 10: How much of productivity differences come from inefficiency?
\smallskip
\item \bibentry{bloom2010management}
\end{itemize}
\bigskip

\noindent \textbf{\textit{Week 8: Catch-up and Review for Midterm}}

\noindent \textbf{\textit{Week 9: Midterm Exam}}

\noindent \textbf{\textit{Week 10: Government}}
\begin{itemize}
\item Weil Chapter 12: Government
\smallskip
\item Pages 28 to 32 of \bibentry{griffiths2015economist}
\smallskip
\item \bibentry{johnson2017states}
\smallskip
\item Tyler Cowen \href{https://marginalrevolution.com/marginalrevolution/2020/01/what-libertarianism-has-become-and-will-become-state-capacity-libertarianism.html}{``What libertarianism has become and will become---State Capacity Libertarianism''}
\smallskip 
\item \bibentry{guriev2019gorbachev}
\smallskip
\item \textbf{Problem Set 2 Collected}
\end{itemize}
\bigskip

\noindent \textbf{\textit{Week 11: Inequality}}
\begin{itemize}
\item Weil Chapter 13: Income Inequality
\smallskip
\item \bibentry{ravallion2018inequality}
\smallskip
\item \bibentry{easterly2019review}
\end{itemize}
\bigskip

\noindent \textbf{\textit{Week 12: Culture}}
\begin{itemize}
\item Weil Chapter 14: Culture
\smallskip
\item \bibentry{guiso2006does}
\smallskip
\item \bibentry{hoff2014making}
\end{itemize}
\bigskip

\noindent \textbf{\textit{Week 13: Geography, Climate, and Natural Resources}}
\begin{itemize}
\item Weil Chapter 15: Geography, Climate, and Natural Resources
\smallskip
\item \bibentry{anderson2016jewish}
\smallskip
\item \bibentry{nunn2010columbian}
\end{itemize}
\bigskip

\noindent \textbf{\textit{Week 14: Review of Material for Final Exam}}
\bigskip

\newpage

{
\bibliographystyle{apalike}\bibliography{noelref.bib}}
\end{document}



