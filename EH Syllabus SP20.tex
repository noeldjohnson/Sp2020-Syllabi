\documentclass[11pt]{article}
\usepackage{graphicx}
\usepackage{float}
\usepackage{mathrsfs}
\usepackage{amsfonts}
\usepackage{footmisc}
\usepackage{amsmath}
\usepackage{amssymb}
\usepackage{parskip}
\usepackage{appendix}
\usepackage{natbib}\setlength{\bibsep}{0.2pt}
\usepackage{caption}
\usepackage{fullpage}
\usepackage{setspace}
\usepackage{bibentry}
\usepackage{enumerate}
\usepackage{hyperref} 
\pagestyle{empty}
\renewcommand{\thefootnote}{\fnsymbol{footnote}}

\begin{document}
\nobibliography*

\begin{center}
{\bf Econ. 365 Topics in Economic History \\
George Mason University, Spring 2020 \\
W 4:30 -- 7:10,  Room:  Robinson Hall B111}
\end{center}

\setlength{\unitlength}{1in}

\begin{picture}(6,.1) 
\put(0,0) {\line(1,0){6.25}}         
\end{picture}

 

\renewcommand{\arraystretch}{2}

\vskip.15in
\noindent\textbf{Instructor:} Noel D. Johnson
\vskip.15in
\noindent\textbf{Email:} \href{mailto:noeldjohnson@mac.com}{noeldjohnson@mac.com}
\vskip.15in
\noindent\textbf{Office Location:} Carow 8
\vskip.15in
\noindent\textbf{Office Hours:} TBA
\vskip.15in
\noindent\textbf{Course Webpage:}  \url{https://noeldjohnson.github.io/eeh/}

\vskip.15in
\noindent \textbf{Course Description}

In this course we will study history through the lens of economic theory.  The course will cover subjects beginning around the medieval period and extending to the post-colonial era.  The main focus will be on Western Europe between 1300 and 1800, though we will also draw comparisons with the Middle East and China.  The three central topics which tie the course together are:  (i)   The mutability of different forms of social identity and political organization;  (ii)  The rise of the western part of Eurasia (Europe) to economic dominance by the onset of the Age of Exploration in the 1500's;  (iii)  An understanding of the potential causes and consequences of the Great Divergence in incomes around 1800. 


\vskip.15in
\noindent\textbf{Grades}

\vskip.15in
\textit{Eighty percent of success is showing up} -- Woody Allen \\
\vskip.15in


Midterm Exam worth 30\% of your final grade.

Comprehensive Final Exam worth 30\% of your final grade.

Book essay worth 30\% of your final grade.

Class attendance and participation is 10\% of your final grade.

\vskip.15in
\noindent\textbf{A Note on Exams}

There will be two exams in this course, one midterm and a final. The final will be administered during the University assigned testing period and is comprehensive.  You are required to take the final, failure to do so will result in an F for the course.  The final exam will only be rescheduled for a University condoned reason and I will ask for a note from your doctor (and will call said to doctor to confirm the note) if this is your excuse.

The timing of the midterm is placed approximately in the course outline below.  If you miss the midterm, then I will not give a make-up exam.  Instead, I will increase the value of your final exam to 60\% of your course grade.

Exams will consist of a mix of short definitions and essay questions.  They will focus on the readings and class lecture.

If you wish to dispute the grading of an exam you must submit a typed request making explicit reference to the problem(s) along with the original test within two class periods (one week). I will then review your arguments and decide if a mistake was made. Under no circumstances will I discuss grading with a student until after I have received and reviewed the written complaint. Office hours are for me to help you understand course material, not for grade disputes.

\vskip.35in
\noindent\textbf{Grade Disputes}

If you wish to dispute the grading of an exam you must submit a typed request making explicit reference to the problem(s) along with the original test within two class periods (one week) of the day I hand back the graded exams. I will then review your arguments and decide if a mistake was made. Under no circumstances will I discuss grading with a student until after I have received and reviewed the written complaint. Office hours are for me to help you understand course material, not for grade disputes.

\vskip.35in
\noindent\textbf{Book Essay}

This course fulfills the Writing Intensive requirement in the Economics major.  It does so through the completion of a book essay worth 30\% of your grade. You will be assigned two books from the list below at random. You will write an essay describing how these books relate to the themes we discussed in class. The paper should start with an introduction that lays out what books you have chosen and briefly describes their main hypotheses. You should then spend a significant portion of the paper overviewing the main arguments of the books. Please explain what the arguments are and how the author(s) substantiate the arguments. How do these arguments relate to the theory or other readings we've done in class? Do they complement each other or are they critiques?

The paper should be typed in a reasonable font. Double spaced with reasonable margins. The paper should be around 15 pages. The paper is due on the last day of class---6 May---by midnight. \textbf{I will only accept pdf's emailed to me}. For every 24 hours the paper is late, your paper grade will be reduced by one half of a letter grade (i.e. 5\%).
\bigskip


\begin{enumerate}
\item \bibentry{kuran2012long}
\smallskip
\item \bibentry{mccloskey2010bourgeois}
\smallskip
\item \bibentry{pomeranz2009great}
\smallskip
\item \bibentry{hoffman2017did}
\smallskip
\item \bibentry{scheidel2018great}
\smallskip
\item \bibentry{allen2009british}
\smallskip
\item \bibentry{mokyr2010enlightened}
\smallskip
\item \bibentry{rubin2017rulers}
\smallskip
\item \bibentry{johnson2019persecution}
\smallskip
\item \bibentry{acemoglu2019narrow}
\smallskip
\item \bibentry{north2009violence}
\smallskip
\item \bibentry{diamond1998guns}
\end{enumerate}


\vskip.35in
\noindent\textbf{Class Attendance and Participation}

Participation is important for this class and you should attend class unless you are ill. I will hand out sign-up sheets on randomly chosen dates to evaluate attendance. You can miss one class for any reason. Beyond one class, each recorded absence will negatively impact your final grade by 2.5-percentage points.

\vskip.15in
\noindent\textbf{The Timeline}

Attached to this syllabus is a timeline of events I've put together to help you keep your bearings in the course.  I expect everyone to have the dates and events on this timeline \textbf{memorized by the midterm exam}.  No exceptions.  This is a rote exercise, however, it is a useful one nonetheless.  Unless you have some context for the theories and trends we discuss, you'll be lost.  I suggest you sit down in front of a computer with internet access and systematically go through each event by visiting the relevant wikipedia page.  This will give you more background on the entries and, as a bonus, you'll find that these are actually interesting events.  I will test you on these dates and events on the midterm and the final.  Furthermore, if I find the class is not exhibiting sufficient interest/attendance with regards to the lectures, I will start giving out pop quizzes which cover the timeline (as well as other material from lecture).  These quizzes will count towards your participation grade in a manner of my own deciding.

\vskip.15in
\noindent\textbf{Course Materials}

There are no books required for this course.

We will be reading many articles which are available for download on the class's Dropbox folder accessed via a link on the course webpage.  In addition, most of the readings can be downloaded from either JSTOR, the NBER Working Papers archive, or EconLit through the GMU libraries research databases page.  You are required to acquire these papers and read them before the class in which they are covered.  You are expected to know the readings for the exams.

If you would like more material on the empirical methods we cover an excellent source is:

\begin{itemize}

\item Joshua Angrist and Jorn-Steffen Pischke, Mastering Metrics:  The Path from Cause to Effect. (Princeton University Press, 2014).

\end{itemize}

A good place to get advice on how to write a paper is here:

\begin{itemize}

\item \bibentry{Mccloskey00}

\end{itemize}

Or, you could take a look at the material on my webpage here:  \url{https://noeldjohnson.github.io/student_advice/}

We will also be reading articles which are available for download on a Dropbox folder accessed via a link on the course webpage.  In addition, most of the readings can be downloaded from either JSTOR, the NBER Working Papers archive, or EconLit through the GMU libraries research databases page.  You are required to acquire these papers and read them before the week they are listed on the syllabus. You are expected to know the readings for the exams.

\vskip.15in
\noindent\textbf{Acknowledgements}

A lot of the material used for this course, especially the lecture slides, come from \href{https://www.jaredcrubin.com/home}{Jared Rubin} and are used with his permission.

\vskip.15in
\noindent\textbf{Some Important Dates}

First Day of Classes:  22 January

Spring Break: 9 March to 15 March

Last Day of Classes:  29 April

Final Exam: Wed. 6 May from 4:30 pm--7:15 pm

\newpage
\noindent\textbf{Timeline}
\vspace{.25in}

\begin{description}

\item[c.a. 8,000 BC] Settled agriculture established in the Fertile Crescent (part of the Neolithic Revolution).

\item[44 BC]  Julius Ceaser, conquerer of Gaul, stabbed to death in the Roman Senate.  Transition from Roman Republic to the Roman Empire.

\item[390 BC]  Rome is sacked after the Battle of Allia by the Gauls.  Will be sacked again in 410 (Visigoths), 455 (Vandals), 546 (Ostrogoths), 846 (Arabs, or, `Saracens'), 1084 (Normans), and 1527 (Charles V, Holy Roman Emperor).

\item[622]  The Hijra.  Mohammed and his followers go to Medina to escape persecution.  Mecca conquered by 630.  Shiites and Sunnis split over succession issues after Mohammed's death in 632.

\item[800]  Charlemagne crowned Holy Roman Emperor by Pope Leo III.  Unifies rule over most of Western Europe.

\item[843]  Treaty of Verdun.  Divided the Carolingian Empire into three kingdoms (west, middle, and east Francia) between the three surviving grandsons of Charlemagne.  

\item[1066] William the Conquerer (from Normandy) defeats English forces at the Battle of Hastings.

\item[1075] Pope Gregory VII issues the \textit{Dictatus Papae} which sparks the Investiture Controversy between himself and Henry IV, Holy Roman Emperor.  Henry forced to abdicate and Germany remains fractionalized for centuries to come.

\item[1209--1255]  Albigensian Crusade fought by the Catholic Church against ``Catharism'' in the region of Languedoc.

\item[1215]  Magna Carta.  Requires English monarch to recognize certain liberties of the people.

\item[1241]  \"{O}gedei Khan dies.  Mongol armies recalled from eastern edge of Europe.

\item[1337--1453]  Hundred Years War.  Fought between kings of France and England for control of French throne.  French forces emerge victorious.  Civil conflict known as the War of Roses commences in England.

\item[1347--1352]  Bubonic Plague (the Black Death) sweeps through Europe.  Kills between thirty and sixty percent of population.

\item[1440's]  Johannes Gutenberg perfects his printing press in Mainz.  Significantly lowers the cost of the printed word. 

\item[1453]  Constantinople is sacked by Ottoman Emperor Mehmed II.

\item[1486]  Pico della Mirandola publishes the \textit{Oration on the Dignity of Man}.  Argues for the shifting of intellectual focus away from Heaven towards Man.  The ``Manifesto of the Renaissance''.  Pope declares his writings ``partly heretical''.

\item[1492]  Columbus ``discovers'' the Americas.  Trip is financed by Ferdinand and Isabella of Spain (as well as Italian financiers). 

\item[1517]  Luther posts 95 Theses on Church door in Wittenburg.  Marks the beginning of the Reformation.

\item[1532]  Collision at Cajamarca between Incan Emperor Atahuallpa and the Spanish conquistador Francisco Pizarro.  80,000 Incans defeated by 168 Spanish soldiers.   

\item[1539]  Edict of Villers Cotter\^{e}ts issued by Francis I.  Important step in the legal centralization of French state.  French becomes official language of law.  Marriage required to be registered with state by priest and registered by a notary.

\item[1543]  Copernicus publishes \textit{De revolutionibus orbium coelestium}.  Advances the heliocentric model of the universe.

\item[1555]  Peace of Augsburg.  Within the Holy Roman Empire state religion is determined by the dictum \textit{cuius regio, eius religio}. 

\item[1581]  The Dutch Republic is founded.

\item[1598]  Edict of Nantes issued by Henri IV of France.  Ends the Wars of Religion by separating civil from religious unity and recognizing rights of Protestants (Huguenots).

\item[1610]  Galileo Gallilee publishes his observations of the phases of heavenly bodies \textit{Sidereus Nuncius}.  Eventually found `gravely suspect of heresy' by Roman Inquisition and sentenced to indefinite house arrest in 1633.

\item[1618--1648]  Thirty Years' War.  Begins as conflict between Catholics and Protestants in Holy Roman Empire.  Finishes as a great powers conflict (Habsburg vs. Bourbon).  Population drops between twenty-five and forty percent within German states.

\item[1648--1653]  The Fronde in France.  Civil conflict between nobility and the crown.  King wins.  Assumes power as Louis XIV (`The Sun King') in 1661.

\item[1649]  Charles I of England beheaded.  Monarch overthrown.  Oliver Cromwell eventually becomes Lord Protector.  Monarchy Restored in 1660.  Cromwell's body exhumed and beheaded.  

\item[1651]  Thomas Hobbes publishes \textit{Leviathan}.  Early example of social contract theory.  Apologia for absolute monarchy.

\item[1654]  Otto von Guericke invents the first vacuum pump.

\item[1685]  Revocation of the Edict of Nantes by Louis XIV.  Up to 40,000 Protestants leave France.  Within a generation the Revocation is left largely unenforced by French authorities.

\item[1687]  Isaac Newton publishes the \textit{Principia}.  ``Nature, and Nature's laws lay hid in night:  God said, `Let Newton Be!' and all was light.'' -- Alexander Pope

\item[1688]  Glorious Revolution.  William III of Orange joins with Mary II of England to overthrow King James II of England.  Monarchy becomes subsidiary to Parliament.

\item[1712]  Newcomen steam engine invented.  Works based on vacuum pressure.  Pumps water out of coal mines thereby lowering the cost of energy necessary for Industrialization.

\item[1756--1763]  Seven Years War.  Great powers conflict between Great Britain and Bourbons involving most of Europe.  Most economically costly conflict for participating governments up to that time.

\item[1776]  Adam Smith publishes the \textit{Wealth of Nations}.  Core text of classical economics.

\item[1789] French Revolution.  French absolute monarchy is over-thrown.  Louis XVI beheaded in 1793.  Europe descends into war.

\item[1790]  Ratification of the U.S. Constitution.  Rhode Island is last state to sign.

\item[c.a. 1800]  The Great Divergence in incomes begins, more or less, around this time.  Income per capita of present-day rich countries increases by a factor of twelve between 1800 and 2000.  Gap between richest and poorest countries rises to a factor of fifty.

\item[1812]  Napoleon invades Russia.  Russians engage in scorched earth warfare.  French lose 360,000 of 400,000 troops by 1813.

\item[1832]  Great Reform Act passed by British Parliament.  Franchise extended to about one of every six adult males.  Major redistricting to account for urban growth in wake of Industrial Revolution.    

\item[1848]  Year of Revolutions in Europe.  Over fifty countries experience significant social unrest.  Demands for more participatory democracy among other issues.  Reactionaries in control within a year.

\item[1859]  Darwin publishes the \textit{On the Origin of Species}.  Lays out principles and evidence for evolution.

\item[1861]  Italy is unified (more or less).

\item[1861--1865]  U.S. Civil War fought between the North and the South.  Abraham Lincoln issues the Emancipation Proclamation on 1 January 1863. 

\item[1870-1871]  Franco--Prussian war.  Paris placed under siege by Prussia.  France capitulates.  Left-wing Parisian Commune repressed at cost of 20,000 lives.  Third Republic declared.   

\item[1871]  Germany is unified.

\item[1914 - 1918]  World War I.  Global war.  Seventeen million die.  Colonial powers begin to break up.  In 1917 Communist Revolution begins in Russia.  Influenza pandemic kills between 50 and 130 million in the immediate aftermath of war.

\item[1916]  Einstein publishes the General Theory of Relativity.  Speed of light is recognized as the speed limit of the universe.

\item[1929]  Crash of U.S. Stock Exchange.  Beginning of the Great Depression.  Deflationary spiral in 1931 exacerbates the situation.

\item[1939-1945]  World War II.  Between 50 and 70 million die, including between 11 and 17 million jews, gypsies, criminals, and intellectuals in Axis concentration camps.  Strategic bombing of civilian targets used by both sides.  100,000 burned to death by Allied bombing in Tokyo in two days.  90,000 to 166,000 killed in the atomic bombing of Hiroshima in 1945.

\item[1954]  Alan Turing commits suicide.  Inventor of the Turing Machine;  a model of a general purpose computer.

\end{description}

\newpage
\vskip.15in
\noindent\textbf{PLEASE NOTE: COURSE POLICIES}
 
\textbf{1. George Mason University Honor System and Code}

\textit{Honor Code}

George Mason University has an Honor Code, which requires all members of this community to maintain the highest standards of academic honesty and integrity. Cheating, plagiarism, lying, and stealing are all prohibited.
 
All violations of the Honor Code will be reported to the Honor Committee.

Plagiarism (statements from Mason Web Site)

Plagiarism means using the exact words, opinions, or factual information from another person without giving that person credit.

http://mason.gmu.edu/~montecin/plagiarism.htm\#plagiarism
 
Please familiarize yourself with the Honor System and Code, as stated in the George Mason University Undergraduate Catalog. When you are given an assignment as an individual, the work must be your own. Some of your work may be collaborative; source material for group projects and work of individual group members must be carefully documented for individual contributions.

http://mason.gmu.edu/~montecin/plagiarism.htm
 
\textbf{2. Class Registration}
 
Students are responsible for verifying the accuracy of their own schedules. Students need to check PatriotWeb regularly to verify that they are registered for the classes that they think they are. This is particularly important since students are no longer dropped for nonpayment.
 Faculty may not allow a student who is not registered to continue to attend class and may not grade the work of students who do not appear on the official class roster.

Deadlines each semester are published in the Schedule of Classes available from the Registrar's Web Site registrar.gmu.edu

After the last day to drop a class, withdrawing from this class requires the approval of the dean and is only allowed for nonacademic reasons.
 Undergraduate students may choose to exercise a selective withdrawal. See the Schedule of Classes for selective withdrawal procedures.

\textbf{3. Accommodations for students with disabilities:}
 
If you are a student with a disability and you need academic accommodations, please see me and contact the Office of Disability Resources at 703-993-2474. All academic accommodations must be arranged through that office.
 
The need for accommodations should be identified at the beginning of the semester and the specific accommodation has to be arranged through the Office of Disability Resources. Faculty cannot provide accommodations to students on their own (e.g. allowing a student extra time to complete an exam because the student reports having a disability).

\newpage

\vskip.15in
\noindent \textbf{Course Outline} 

This is a very tentative schedule and is subject to change at any time.

* Optional Readings---readings that are referenced in the lectures, but are not at the center of the
lecture. Students interested in the topics should consult these readings.

** Podcasts---from the Tides of History podcast available at \url{https://wondery.com/shows/tides-ofhistory/}. These give an informative and entertaining background of the history covered in the section. Highly recommended for helping form term paper topics and strengthening essay answers on your exams.

\vskip.15in
\textbf{\textit{Week 1:  Introduction and Course Themes}}
\begin{itemize}
\item Chapter 1, ``The Sixteen Page Economic History of the World'' in \bibentry{clark2008farewell}
\item Pages 291 to 300 of \bibentry{Freedman89}
\item \bibentry{nunn2009importance}
\item * \bibentry{north1991institutions}
\item * \bibentry{easterlin1981isn}
\item ** Tides of History podcast: Episode 1 ``The Ebb and Flow of History''
\end{itemize}

\vskip.15in
\textbf{\textit{Week 2:  Agriculture and Guilds}}
\begin{itemize}
\item \bibentry{north1971rise}
\item \bibentry{mccloskey2001english}
\item * \bibentry{richardson2005prudent}
\item \bibentry{epstein1998craft}
\item \bibentry{richardson2005craft}
\item * \bibentry{de2017clans}
\item ** Tides of History podcast: Episode 31 ``Peasants and the Medieval Countryside''
\end{itemize}

\vskip.15in
\textbf{\textit{Week 3:  The Black Death and It's Consequences}}
\begin{itemize}
\item Chapter 2, ``The Logic of the Malthusian Economy'' in \bibentry{Clark07a}
\item \bibentry{JedwabJohnsonKoyamaPPP}
\item \bibentry{jedwab2019negative}
\item \bibentry{voigtlander2013gifts}
\item ** Tides of History podcast: Episode 25 ``The Black Death''
\end{itemize}

\vskip.15in
\textbf{\textit{Week 4:  Trade and Exchange}}
\begin{itemize}
\item \bibentry{greif2000fundamental}
\item \bibentry{greif1989reputation}
\item \bibentry{schulz2019kin}
%\item Avner Greif, Paul R. Milgrom, and Barry R. Weingast (1994). ?Coordination, Commitment and Enforcement: The Case of the Merchant Guild, Journal of Political Economy.
\item \bibentry{greif2006history}
\item \bibentry{ogilvie2014institutions}
\item ** Tides of History podcast: Episode 9 ``The Rise of Capitalism and the Early Modern Economic Explosion''
\item ** Tides of History podcast: Episode 10 ``Big Business, Small Business, and the Rise of Capitalism''
\end{itemize}

\vskip.15in
\textbf{\textit{Week 5:  Law, Exchange, and Comparison with the Middle East}}
\begin{itemize}
\item \bibentry{kuran2005absence}
\item \bibentry{balla2009fiscal}
\item \bibentry{Jha2013}
\item \bibentry{JohnsonKoyama16}
\item ** Tides of History podcast: Episode 35 ``Holy War and the Rise of the Ottomans''
\end{itemize}

\vskip.15in
\textbf{\textit{Week 6:  Technological Change and the Reformation}}
\begin{itemize}
\item \bibentry{buringh2009charting}
\item \bibentry{Dittmar11}
\item \bibentry{becker2016causes}
\item \bibentry{iyigun2008luther}
\item \bibentry{borner2019time}
\item ** Tides of History podcast: Episode 18 ``Gutenberg and the Printing Press''
\item ** Tides of History podcast: Episode 19 ``The Printing Press and the Information Revolution''
\item ** Tides of History podcast: Episode 22 ``Martin Luther and the Early Reformation''
\item ** Tides of History podcast: Episode 23 ``John Calvin, Henry VIII, and the Counter-Reformation''
\end{itemize}

\vskip.15in
\textbf{\textit{Week 7:  Institutional Change in Western Europe}}
\begin{itemize}
\item * \bibentry{cantoni2018religious}
\item \bibentry{van2012rise}
\item \bibentry{north1989constitutions}
\item \bibentry{pincus2011really}
\item \bibentry{Lamoreaux11}
\item * \bibentry{jha2015financial}
\item ** Tides of History podcast: Episode 39 ``The Wars of the Roses''
\end{itemize}

\vskip.15in
\textbf{\textit{Week 8:  Catch-up and Review}}

\vskip.15in
\textbf{\textit{Week 9:  Midterm Exam}}

\vskip.15in
\textbf{\textit{Week 10:  The Rise of Cities and the State}}
\begin{itemize}
\item \bibentry{bosker2013baghdad}
\item \bibentry{blaydes2013feudal}
\item \bibentry{dincecco2009fiscal}
%\item * Mark Dincecco (2015). ?The Rise of Effective States in Europe,? Journal of Economic History.
\item \bibentry{JohnsonKoyamaStateCapacity}
\item ** Tides of History podcast: Episode 2 ``The Rise of the State''
\item ** Tides of History podcast: Episode 5 ``The Military Revolution, 1350-1650''
\end{itemize}

\vskip.15in
\textbf{\textit{Week 11:  Colonization}}
\begin{itemize}
\item \bibentry{AcemogluEtAl01}
\item * \bibentry{Dell10}
\item * \bibentry{valencia2018mission}
\item \bibentry{banerjee2005history}
\item \bibentry{Nunn08}
\item * \bibentry{whatley2018gun}
\item \bibentry{michalopoulos2016long}
\item ** Tides of History podcast: Episode 20 ``Medieval Exploration and the Age of Discovery''
\item ** Tides of History podcast: Episode 21 ``Columbus, da Gama, and the Age of Discovery''
\end{itemize}

\vskip.15in
\textbf{\textit{Week 12:  Technology and the Industrial Revolution}}
\begin{itemize}
\item \bibentry{mokyr2018past}
\item \bibentry{allen2011industrial}
\item \bibentry{kelly2016adam}
\item * \bibentry{allen2009british}
\end{itemize}

\vskip.15in
\textbf{\textit{Week 13:  Consequences of Industrialization: Income Distribution and Wages}}
\begin{itemize}
\item \bibentry{lindert1986unequal}
\item \bibentry{piketty2014inequality}
\item * \bibentry{hoffman2002real}
\item \bibentry{allen2001great}
\item \bibentry{allen2011wages}
\item \bibentry{broadberry2018china}
\end{itemize}

\vskip.15in
\textbf{\textit{Week 14:  In Class Review}}


\newpage

\bibliographystyle{apalike}
\bibliography{noelrefHist.bib,noelrefAEH.bib,noelrefDev.bib}

\end{document}





